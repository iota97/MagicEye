\documentclass[12pt, a4paper]{article}
\usepackage[utf8]{inputenc}
\usepackage[english]{babel}
\usepackage{fancyhdr}
\usepackage{datetime}
\usepackage{hyperref}
\usepackage{longtable}
\usepackage{graphicx}
\usepackage[font=small,labelfont=bf]{caption}
\graphicspath{ {Images/} }
\hypersetup{
    colorlinks,
    citecolor=black,
    filecolor=black,
    linkcolor=black,
    urlcolor=black
}

\def\labelitemi{--}
\setcounter{tocdepth}{3}
\pagestyle{fancy}
\fancyhf{}
\renewcommand{\headrulewidth}{1pt}
\renewcommand{\footrulewidth}{1pt}
\rhead{\leftmark}
\rfoot{Page \thepage}

\begin{document}

\begin{titlepage}

    \newcommand{\HRule}{\rule{\linewidth}{0.5mm}} % Defines a new command for the horizontal lines, change thickness here
    
    \center % Center everything on the page
     
    %----------------------------------------------------------------------------------------
    %	HEADING SECTIONS
    %----------------------------------------------------------------------------------------
    
    \textsc{\LARGE Università Degli Studi Di Milano}\\[1.5cm] % Name of your university/college
    \textsc{\Large Real-Time Graphics Programming}\\[0.5cm] % Major heading such as course name
    %\textsc{\large Assignment 1}\\[0.5cm] % Minor heading such as course title
    
    %----------------------------------------------------------------------------------------
    %	TITLE SECTION
    %----------------------------------------------------------------------------------------
    
    \HRule \\[0.4cm]
    { \huge \bfseries Real-time autostereogram rendering pipeline }\\[0.4cm] % Title of your document
    \HRule \\[1.5cm]
     
    %----------------------------------------------------------------------------------------
    %	AUTHOR SECTION
    %----------------------------------------------------------------------------------------
    
    \begin{minipage}{0.4\textwidth}
    \begin{flushleft} \large
    \emph{Author:}\\
    Giovanni \textsc{Cocco} \\
    \end{flushleft}
    \end{minipage}
    ~
    \begin{minipage}{0.4\textwidth}
    \begin{flushright} \large
    \emph{Academic year:} \\
    2022-2023\\
    \end{flushright}
    \end{minipage}\\[2cm]
    
    % If you don't want a supervisor, uncomment the two lines below and remove the section above
    %\Large \emph{Author:}\\
    %John \textsc{Smith}\\[3cm] % Your name
    
    %----------------------------------------------------------------------------------------
    %	DATE SECTION
    %----------------------------------------------------------------------------------------
    
    {\large \today}\\[4cm] % Date, change the \today to a set date if you want to be precise
    
    %----------------------------------------------------------------------------------------
    %	LOGO SECTION
    %----------------------------------------------------------------------------------------
    
    \includegraphics[width=130px, keepaspectratio]{img/unimi.png}\\[1cm] % Include a department/university logo - this will require the graphicx package
     
    %----------------------------------------------------------------------------------------
    
    \vfill % Fill the rest of the page with whitespace
    
    \end{titlepage}

\clearpage
\tableofcontents{}
\listoffigures
\listoftables
\clearpage

\section{Abstract}
In this document we discuss the algorithms, the choices and the implementation details of a real-time autostereogram rendering pipeline built with OpenGL 4.3.\\\\
The pipeline presented allow the rendering of autostereograms starting from a depth buffer that is always produces
as a result on all classical rasterization rendering algorithms. More elaborated effect can be achieved if we supply as input also the color buffer and 
the normal buffer.\\\\
We will first explain what an autostereogram is and how it works from a psychological point of view. We will then discuss a general algorithm to create them.
We will then discuss the implementation detail of the project.\\\\
It should be noted that the input needed for the autostereogram generation algorithm are produced as a result of most real-time rendering pipeline and this 
mean that is possible to apply this technique to an already existent rendering pipeline.\\\\
For this project we made a simple rendering pipeline using Blinn-Phong illumination model, as we will see more sophisticated model give little gain in 
the final result.\\\\
The project also implements blend skinning for skeletal animation, we will discuss the implementation and the difference between linear blend skinning and
dual quaternion skinning.\\\\
We will see how the performance of this pipeline can be totally in the frame budget on most GPU buy may suffer on old laptop one while still
providing 60 frames per second.

\section{Autostereogram}
\subsection{What is an autostereogram}

\subsection{How to see an autostereogram}

\subsection{Autostereogram rendering algorithm}

\subsection{Advanced effects}

\section{Scene management}
\subsection{Handle multiple scenes}

\subsection{Assets loading}

\subsection{Object struct}

\subsection{Updating the scene}

\subsection{Camera and input processing}

\section{GUI}

\section{Rendering}
\subsection{Illumination model}

\subsection{Toon shading}

\subsection{Skybox}

\subsection{Autostereogram}

\section{Skeletal animation}

\subsection{Linear blend skinning}

\subsection{Dual quaternion skinning}

\section{Performance evaluation}

\section{Conclusion}

\end{document}